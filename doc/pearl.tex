\documentclass{article}

\usepackage{todonotes}

\usepackage[T1]{fontenc}
\usepackage[latin1]{inputenc}

% grab parts of other files
\usepackage{catchfilebetweentags}
\makeatletter

\newrobustcmd*\OrigExecuteMetaData[2][\jobname]{%
\CatchFileBetweenTags\CatchFBT@tok{#1}{#2}%
\global\expandafter\CatchFBT@tok\expandafter{%
\expandafter}\the\CatchFBT@tok
}%\OrigExecuteMetaData

\newrobustcmd*\ChkExecuteMetaData[2][\jobname]{%
\CatchFileBetweenTags\CatchFBT@tok{#1}{#2}%
\edef\mytokens{\detokenize\expandafter{\the\CatchFBT@tok}}
\ifx\mytokens\empty\PackageError{catchfilebetweentags}{the tag #2 is not found\MessageBreak in file #1 \MessageBreak called from \jobname.tex}{use a different tag}\fi%
}%\ChkExecuteMetaData

\renewrobustcmd*\ExecuteMetaData[2][\jobname]{%
\ChkExecuteMetaData[#1]{#2}%
\OrigExecuteMetaData[#1]{#2}%
}

\makeatother


% tree diagrams
\usepackage{tikz}
\usetikzlibrary{matrix}

% algorithm
\usepackage[linesnumbered,ruled,vlined]{algorithm2e}


\usepackage{cleveref}

\title{Certified Binary Search in a Read-Only Array}
\author{Guillaume ALLAIS}

\begin{document}
\maketitle

\begin{abstract}
  Reifying the control flow of imperative programs as data structures allows
  functional programmers to implement common algorithms by an obviously
  terminating structural recursion. Additionally, this inductive structure
  can be decorated with invariants allowing the procedure to be proven correct.

  This ease of definition and certification comes at a cost because the
  functional data structure is typically not stored as efficiently in
  memory as the imperative counterpart it replaces.

  In this paper we use quantitative type theory as implemented in Idris 2 to
  combine an efficient runtime representation of the data with a convenient
  compile time inductive view of it to use for ease of definition and
  certification.
\end{abstract}

\section{Introduction}

\subsection{Binary search, by example}

\begin{figure}
  \center
  \ExecuteMetaData[tree.tex]{search11}
  \caption{Tracing the binary search for 11}
\end{figure}

\begin{figure}
  \center
  \ExecuteMetaData[tree.tex]{search31}
  \caption{Tracing the binary search for 31}
\end{figure}

\subsection{Binary search, imperatively}

\documentclass{article}

\usepackage{xcolor}
\usepackage[linesnumbered,ruled,vlined]{algorithm2e}

\usepackage[T1]{fontenc}
\usepackage[latin1]{inputenc}

%%% Coloring the comment as blue
\newcommand\mycommfont[1]{\footnotesize\ttfamily\textcolor{blue}{#1}}
\SetCommentSty{mycommfont}

\begin{document}

\begin{algorithm}[H]
\SetKwInput{KwInput}{Input}
\SetKwInput{KwOutput}{Output}              % set the Output
\DontPrintSemicolon

  \KwInput{Needle and array}
  \KwOutput{Is the needle in the array?}

% Set Function Names
  \SetKwFunction{FSearch}{Search}
% Write Function with word ``Function''
  \SetKwProg{Fn}{Procedure}{:}{}
  \Fn{\FSearch{needle, array}}{

        found := false\;
        begin := 0\;
        end   := \texttt{size}(array)\;

        \;
        \While{\texttt{not}(found) \&\& begin < end}
          {\;
          middle    := begin + (end - begin) / 2\;
          candidate := array[middle]\;\;

         \If{candidate = needle} {found := true}
         \ElseIf{needle < candidate}{end := middle}
         \Else{begin := middle + 1}\;
        }
         \KwRet{found}\;
  }
\caption{Binary search in a sorted array}
\end{algorithm}

\end{document}


\subsection{The computation's underlying structure}

Using data as a source of control flow is a classic design pattern in functional
programming. It not only serves as a replacement for imperative constructs, it
provides the user with a more general abstraction. This is perhaps best exemplified
by the common practice of \emph{traversing}\todo{cite}
lists instead of using a \texttt{for}-loops. Any list-producing process (a function,
a system call, etc.) can now become a source of control flow.

Coming back to our running example, if we take the time to trace all the possible
execution paths of binary search on our read only array we see a tree revealed
(cf. \cref{fig:searchall}).
This is the underlying structure of the binary search computation: a binary search
tree! \todo{Be precise: AVL?}

\begin{figure}
  \center
  \ExecuteMetaData[tree.tex]{searchall}
  \caption{Tracing all possible binary searches}
  \label{fig:searchall}
\end{figure}

This suggests a straightforward avenue for formalisation: ditch the array and
use the underlying binary search tree instead. This has the added benefit of
replacing an induction over the size of the subarray the search is performed on
into a purely structural one on the tree.

However this would have adverse consequences in terms of performance: instead of
a compact representation using a contiguous array, we would have a tree whose nodes
comprise of a value and two pointers to its children.

Can we have the best of both worlds? A compact runtime representation using an
array and an obviously terminating recursive function amenable to verification.

\section{}


\end{document}
